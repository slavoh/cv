%%%%%%%%%%%%%%%%%%%%%%%%%%%%%%%%%%%%%%%%%
% Medium Length Graduate Curriculum Vitae
% LaTeX Template
% Version 1.1 (9/12/12)
%
% This template has been downloaded from:
% http://www.LaTeXTemplates.com
%
% Original author:
% Rensselaer Polytechnic Institute (http://www.rpi.edu/dept/arc/training/latex/resumes/)
%
% Important note:
% This template requires the res.cls file to be in the same directory as the
% .tex file. The res.cls file provides the resume style used for structuring the
% document.
%
%%%%%%%%%%%%%%%%%%%%%%%%%%%%%%%%%%%%%%%%%

%----------------------------------------------------------------------------------------
%    PACKAGES AND OTHER DOCUMENT CONFIGURATIONS
%----------------------------------------------------------------------------------------

\documentclass[margin, 10pt]{res} % Use the res.cls style, the font size can be changed to 11pt or 12pt here

\usepackage{helvet} % Default font is the helvetica postscript font
%\usepackage{newcent} % To change the default font to the new century schoolbook postscript font uncomment this line and comment the one above
\usepackage[slovak]{babel}
\usepackage[utf8]{inputenc}
\usepackage[T1]{fontenc}
\usepackage[hidelinks]{hyperref}
\hypersetup{
	colorlinks   = true, %Colours links instead of ugly boxes
	urlcolor     = blue, %Colour for external hyperlinks
	linkcolor    = blue, %Colour of internal links
	citecolor   = red %Colour of citations
}

\setlength{\textwidth}{5.1in} % Text width of the document

\begin{document}
	
	%----------------------------------------------------------------------------------------
	%    NAME AND ADDRESS SECTION
	%----------------------------------------------------------------------------------------
	
	\moveleft.5\hoffset\centerline{\large\bf Slavomír Hanzely} % Your name at the top
	
	\moveleft\hoffset\vbox{\hrule width\resumewidth height 1pt}\smallskip % Horizontal line after name; adjust line thickness by changing the '1pt'
	
	\moveleft.5\hoffset\centerline{shanzely@gmail.com} % Your address
	\moveleft.5\hoffset\centerline{+421 948 555 355}
	
	
	%----------------------------------------------------------------------------------------
	
	\begin{resume}
		
		%----------------------------------------------------------------------------------------
		%    OBJECTIVE SECTION
		%----------------------------------------------------------------------------------------
		
		%----------------------------------------------------------------------------------------
		%    EDUCATION SECTION
		%----------------------------------------------------------------------------------------
		
		\section{EDUCATION}
		
		
		
		\textbf{Faculty of Mathematics, Physics and Informatics, Comenius University},
		Computer Science (3rd-year undergraduate student) \hfill 2016-present \\
		95 credits in 1st year (official max. number - 90, including 2nd, 3rd a master's courses)% average mark 1.4 \footnote{it was spoiled by grade E at 5th grade course, that had been taught for the first time and I overestimated myself}
		\\70 credits in the 2nd year (not including two unofficially passed master's courses in Computer Science and one unofficially passed master's course in Mathematics)\\
		Passed all bachelor finals in advance and with best grades (only bachelor thesis remains to be done).\\
		This semester I am unofficially attending (due to the high amount of credits) 10 courses\footnote{Category Theory, Graph Theory, Combinatorial Structures, Markov Processes, Probability Theory, Selected Topics in Data Structures, Selected Topics in Algebra, Matrix Calculus, Mathematical Analysis(3),  Unstructured Talks on Structures: Chapters in Mathematics for Computer Scientist}: 3 bachelor's courses in Mathematics, 5 master's courses in Computer Science and 2 master's courses in Mathematics.\footnote{Master's courses that I had already passed: Cryptology, Programming Languages, Probabilistic Methods.}
		
		
		
		\textbf{Gymnázium Jána Adama Raymana, Prešov  
		}\hfill 2013-2016\\
		Graduation (Maturita) in 6 subjects (only 4 are compulsory), passed with best grades
		
		\textbf{Coursera} (during high school)  \\
		Machine learning, Calculus (1, 2), Learning how to learn, Introduction to Psychology
		
		%----------------------------------------------------------------------------------------
		%    COMPUTER SKILLS SECTION
		%----------------------------------------------------------------------------------------
		\section{ACHIEVEMENTS}
		\href{http://vjimc.osu.cz/j27/j27results1.html}{Vojtěch Jarník Competition}
		\begin{itemize} \itemsep -2pt % Reduce space between items
			\item[2017:] \textbf{8-10th place} in category 1 (first place within Czech and Slovak contestants)
		\end{itemize}
		
		\href{https://skmo.sk/poradia.php?podlink=ucastnici&uid=7082}{ Mathematical Olympiad}
		\begin{itemize} \itemsep -2pt % Reduce space between items
			\item[2016:] 
			\textbf{3rd place} on the national round, category A (winner) \\
			Participation on \textbf{International Mathematical Olympiad} (IMO) \\
			1st place on regional round, category A
			\item[2015:]  18-20th place (\textbf{bronze medal}) on Middle European Mathematical Olympiad (MEMO)
			\item[2014:]  1st place on regional round, category B
			\item[2013:]  1st place on regional round, category C \\
			Participation on Czech-Polish-Slovak match junior (CPSJ) - 3rd place within Slovak contestants
		\end{itemize}
		\href{http://oi.sk/archive.php}{Olympiad in Computer Science}
		\begin{itemize} \itemsep -2pt % Reduce space between items
			\item[2016:] \ 1st place on regional round, category A
			\item[2015:] \ 2nd place on regional round, category B
		\end{itemize}
		
		
		%----------------------------------------------------------------------------------------
		%    PROFESSIONAL EXPERIENCE SECTION
		%----------------------------------------------------------------------------------------
		
		\section{WORKING EXPERIENCES}
		\textbf{Mathematical Olympiad}
		\begin{itemize}  \itemsep -2pt % Reduce space between items
			\item marking solutions at the national round of Mathematical Olympiad (twice to this day)
			\item organizing one day at the selection camp for International Mathematical Olympiad - creating problem set and marking the solutions (twice to this day)
		\end{itemize}
		\textbf{Trojsten} - volunteering
		\begin{itemize} \itemsep -2pt % Reduce space between items
			\item marking solutions of the correspondent competitions KMS, KSP and iKS (approximately 600 solutions)
			\item organizing camps for talented high school students in Mathematics and Computer Science - up to this day, I have organized 8 camps (3 of them as the main organizer), next one is being prepared.
			\item lecturing 29 lectures (including a half-day lecture on the camp iKS)
		\end{itemize}
		
		\section{PROJECTS}
		\textbf{Stochastic Algorithms for Convex Feasibility and Optimization with Many Constraints} (Supervisor: \href{https://scholar.google.com/citations?user=pGh242UAAAAJ&hl=en}{
			Peter Richt\'{a}rik})\\
		In this project, I was testing algorithms for solving convex feasibility problem (finding a point sufficiently close to complicated convex set) and SAGA algorithm for empirical risk minimization (finding a minimum of a complicated function satisfying a large set of constraints).\\
		Also, parameter tuning was included, I was trying to find the best parameters in practice and compare them to the optimal parameters provided by the theory.\\
		I visited \href{https://www.kaust.edu.sa/en}{King Abdullah University of Science and Technology}) for 39 days in order to work on this project.\\
		You can find a project report \href{https://github.com/slavoh/rocnikac/blob/master/report_new/report.pdf}{here}.
		
		%----------------------------------------------------------------------------------------
		%    COMMUNITY SERVICE SECTION
		%---------------------------------------------------------------------------------------- 
		
		\section{CONFERENCE PARTICIPATION}
		\textbf{Optimization and Big Data conference} (5-7 February 2018)\\
		A conference organized by Peter Richtárik devoted to the most recent advances in Optimization and Machine Learning.
		
		\textbf{ReactiveConf 2017}\\
		Conference devoted to web technologies.
		
		
		\section{SKILLS}
		
		Programming
		\begin{itemize} \itemsep -2pt
			\item advanced in Haskell, C/C++, Julia, Java, Python, \LaTeX, Assembler
			\item basics Octave
			\item Linux
		\end{itemize}
		%----------------------------------------------------------------------------------------
		%    EXTRA-CURRICULAR ACTIVITIES SECTION
		%----------------------------------------------------------------------------------------
		
		\section{HOBBIES} 
		Sport
		\begin{itemize} \itemsep -2pt
			\item ultimate frisbee \\
			participation on European Youth ultimate Championship (in Nederland) - representing Slovak national team\\
			participation on European Youth ultimate Cup (in Hungary) - representing Slovak national team
			\item in past: ice-hockey, floorball, karate
		\end{itemize}
		
		%----------------------------------------------------------------------------------------
		
	\end{resume}
\end{document}